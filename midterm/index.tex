% Options for packages loaded elsewhere
\PassOptionsToPackage{unicode}{hyperref}
\PassOptionsToPackage{hyphens}{url}
\PassOptionsToPackage{dvipsnames,svgnames,x11names}{xcolor}
%
\documentclass[
  letterpaper,
  DIV=11,
  numbers=noendperiod]{scrreprt}

\usepackage{amsmath,amssymb}
\usepackage{iftex}
\ifPDFTeX
  \usepackage[T1]{fontenc}
  \usepackage[utf8]{inputenc}
  \usepackage{textcomp} % provide euro and other symbols
\else % if luatex or xetex
  \usepackage{unicode-math}
  \defaultfontfeatures{Scale=MatchLowercase}
  \defaultfontfeatures[\rmfamily]{Ligatures=TeX,Scale=1}
\fi
\usepackage{lmodern}
\ifPDFTeX\else  
    % xetex/luatex font selection
\fi
% Use upquote if available, for straight quotes in verbatim environments
\IfFileExists{upquote.sty}{\usepackage{upquote}}{}
\IfFileExists{microtype.sty}{% use microtype if available
  \usepackage[]{microtype}
  \UseMicrotypeSet[protrusion]{basicmath} % disable protrusion for tt fonts
}{}
\makeatletter
\@ifundefined{KOMAClassName}{% if non-KOMA class
  \IfFileExists{parskip.sty}{%
    \usepackage{parskip}
  }{% else
    \setlength{\parindent}{0pt}
    \setlength{\parskip}{6pt plus 2pt minus 1pt}}
}{% if KOMA class
  \KOMAoptions{parskip=half}}
\makeatother
\usepackage{xcolor}
\setlength{\emergencystretch}{3em} % prevent overfull lines
\setcounter{secnumdepth}{5}
% Make \paragraph and \subparagraph free-standing
\ifx\paragraph\undefined\else
  \let\oldparagraph\paragraph
  \renewcommand{\paragraph}[1]{\oldparagraph{#1}\mbox{}}
\fi
\ifx\subparagraph\undefined\else
  \let\oldsubparagraph\subparagraph
  \renewcommand{\subparagraph}[1]{\oldsubparagraph{#1}\mbox{}}
\fi

\usepackage{color}
\usepackage{fancyvrb}
\newcommand{\VerbBar}{|}
\newcommand{\VERB}{\Verb[commandchars=\\\{\}]}
\DefineVerbatimEnvironment{Highlighting}{Verbatim}{commandchars=\\\{\}}
% Add ',fontsize=\small' for more characters per line
\usepackage{framed}
\definecolor{shadecolor}{RGB}{241,243,245}
\newenvironment{Shaded}{\begin{snugshade}}{\end{snugshade}}
\newcommand{\AlertTok}[1]{\textcolor[rgb]{0.68,0.00,0.00}{#1}}
\newcommand{\AnnotationTok}[1]{\textcolor[rgb]{0.37,0.37,0.37}{#1}}
\newcommand{\AttributeTok}[1]{\textcolor[rgb]{0.40,0.45,0.13}{#1}}
\newcommand{\BaseNTok}[1]{\textcolor[rgb]{0.68,0.00,0.00}{#1}}
\newcommand{\BuiltInTok}[1]{\textcolor[rgb]{0.00,0.23,0.31}{#1}}
\newcommand{\CharTok}[1]{\textcolor[rgb]{0.13,0.47,0.30}{#1}}
\newcommand{\CommentTok}[1]{\textcolor[rgb]{0.37,0.37,0.37}{#1}}
\newcommand{\CommentVarTok}[1]{\textcolor[rgb]{0.37,0.37,0.37}{\textit{#1}}}
\newcommand{\ConstantTok}[1]{\textcolor[rgb]{0.56,0.35,0.01}{#1}}
\newcommand{\ControlFlowTok}[1]{\textcolor[rgb]{0.00,0.23,0.31}{#1}}
\newcommand{\DataTypeTok}[1]{\textcolor[rgb]{0.68,0.00,0.00}{#1}}
\newcommand{\DecValTok}[1]{\textcolor[rgb]{0.68,0.00,0.00}{#1}}
\newcommand{\DocumentationTok}[1]{\textcolor[rgb]{0.37,0.37,0.37}{\textit{#1}}}
\newcommand{\ErrorTok}[1]{\textcolor[rgb]{0.68,0.00,0.00}{#1}}
\newcommand{\ExtensionTok}[1]{\textcolor[rgb]{0.00,0.23,0.31}{#1}}
\newcommand{\FloatTok}[1]{\textcolor[rgb]{0.68,0.00,0.00}{#1}}
\newcommand{\FunctionTok}[1]{\textcolor[rgb]{0.28,0.35,0.67}{#1}}
\newcommand{\ImportTok}[1]{\textcolor[rgb]{0.00,0.46,0.62}{#1}}
\newcommand{\InformationTok}[1]{\textcolor[rgb]{0.37,0.37,0.37}{#1}}
\newcommand{\KeywordTok}[1]{\textcolor[rgb]{0.00,0.23,0.31}{#1}}
\newcommand{\NormalTok}[1]{\textcolor[rgb]{0.00,0.23,0.31}{#1}}
\newcommand{\OperatorTok}[1]{\textcolor[rgb]{0.37,0.37,0.37}{#1}}
\newcommand{\OtherTok}[1]{\textcolor[rgb]{0.00,0.23,0.31}{#1}}
\newcommand{\PreprocessorTok}[1]{\textcolor[rgb]{0.68,0.00,0.00}{#1}}
\newcommand{\RegionMarkerTok}[1]{\textcolor[rgb]{0.00,0.23,0.31}{#1}}
\newcommand{\SpecialCharTok}[1]{\textcolor[rgb]{0.37,0.37,0.37}{#1}}
\newcommand{\SpecialStringTok}[1]{\textcolor[rgb]{0.13,0.47,0.30}{#1}}
\newcommand{\StringTok}[1]{\textcolor[rgb]{0.13,0.47,0.30}{#1}}
\newcommand{\VariableTok}[1]{\textcolor[rgb]{0.07,0.07,0.07}{#1}}
\newcommand{\VerbatimStringTok}[1]{\textcolor[rgb]{0.13,0.47,0.30}{#1}}
\newcommand{\WarningTok}[1]{\textcolor[rgb]{0.37,0.37,0.37}{\textit{#1}}}

\providecommand{\tightlist}{%
  \setlength{\itemsep}{0pt}\setlength{\parskip}{0pt}}\usepackage{longtable,booktabs,array}
\usepackage{calc} % for calculating minipage widths
% Correct order of tables after \paragraph or \subparagraph
\usepackage{etoolbox}
\makeatletter
\patchcmd\longtable{\par}{\if@noskipsec\mbox{}\fi\par}{}{}
\makeatother
% Allow footnotes in longtable head/foot
\IfFileExists{footnotehyper.sty}{\usepackage{footnotehyper}}{\usepackage{footnote}}
\makesavenoteenv{longtable}
\usepackage{graphicx}
\makeatletter
\def\maxwidth{\ifdim\Gin@nat@width>\linewidth\linewidth\else\Gin@nat@width\fi}
\def\maxheight{\ifdim\Gin@nat@height>\textheight\textheight\else\Gin@nat@height\fi}
\makeatother
% Scale images if necessary, so that they will not overflow the page
% margins by default, and it is still possible to overwrite the defaults
% using explicit options in \includegraphics[width, height, ...]{}
\setkeys{Gin}{width=\maxwidth,height=\maxheight,keepaspectratio}
% Set default figure placement to htbp
\makeatletter
\def\fps@figure{htbp}
\makeatother
\newlength{\cslhangindent}
\setlength{\cslhangindent}{1.5em}
\newlength{\csllabelwidth}
\setlength{\csllabelwidth}{3em}
\newlength{\cslentryspacingunit} % times entry-spacing
\setlength{\cslentryspacingunit}{\parskip}
\newenvironment{CSLReferences}[2] % #1 hanging-ident, #2 entry spacing
 {% don't indent paragraphs
  \setlength{\parindent}{0pt}
  % turn on hanging indent if param 1 is 1
  \ifodd #1
  \let\oldpar\par
  \def\par{\hangindent=\cslhangindent\oldpar}
  \fi
  % set entry spacing
  \setlength{\parskip}{#2\cslentryspacingunit}
 }%
 {}
\usepackage{calc}
\newcommand{\CSLBlock}[1]{#1\hfill\break}
\newcommand{\CSLLeftMargin}[1]{\parbox[t]{\csllabelwidth}{#1}}
\newcommand{\CSLRightInline}[1]{\parbox[t]{\linewidth - \csllabelwidth}{#1}\break}
\newcommand{\CSLIndent}[1]{\hspace{\cslhangindent}#1}

\KOMAoption{captions}{tableheading}
\makeatletter
\makeatother
\makeatletter
\@ifpackageloaded{bookmark}{}{\usepackage{bookmark}}
\makeatother
\makeatletter
\@ifpackageloaded{caption}{}{\usepackage{caption}}
\AtBeginDocument{%
\ifdefined\contentsname
  \renewcommand*\contentsname{Table of contents}
\else
  \newcommand\contentsname{Table of contents}
\fi
\ifdefined\listfigurename
  \renewcommand*\listfigurename{List of Figures}
\else
  \newcommand\listfigurename{List of Figures}
\fi
\ifdefined\listtablename
  \renewcommand*\listtablename{List of Tables}
\else
  \newcommand\listtablename{List of Tables}
\fi
\ifdefined\figurename
  \renewcommand*\figurename{Figure}
\else
  \newcommand\figurename{Figure}
\fi
\ifdefined\tablename
  \renewcommand*\tablename{Table}
\else
  \newcommand\tablename{Table}
\fi
}
\@ifpackageloaded{float}{}{\usepackage{float}}
\floatstyle{ruled}
\@ifundefined{c@chapter}{\newfloat{codelisting}{h}{lop}}{\newfloat{codelisting}{h}{lop}[chapter]}
\floatname{codelisting}{Listing}
\newcommand*\listoflistings{\listof{codelisting}{List of Listings}}
\makeatother
\makeatletter
\@ifpackageloaded{caption}{}{\usepackage{caption}}
\@ifpackageloaded{subcaption}{}{\usepackage{subcaption}}
\makeatother
\makeatletter
\@ifpackageloaded{tcolorbox}{}{\usepackage[skins,breakable]{tcolorbox}}
\makeatother
\makeatletter
\@ifundefined{shadecolor}{\definecolor{shadecolor}{rgb}{.97, .97, .97}}
\makeatother
\makeatletter
\makeatother
\makeatletter
\makeatother
\ifLuaTeX
  \usepackage{selnolig}  % disable illegal ligatures
\fi
\IfFileExists{bookmark.sty}{\usepackage{bookmark}}{\usepackage{hyperref}}
\IfFileExists{xurl.sty}{\usepackage{xurl}}{} % add URL line breaks if available
\urlstyle{same} % disable monospaced font for URLs
\hypersetup{
  pdftitle={DATA 5690: Midterm},
  pdfauthor={Tyler J. Brough},
  colorlinks=true,
  linkcolor={blue},
  filecolor={Maroon},
  citecolor={Blue},
  urlcolor={Blue},
  pdfcreator={LaTeX via pandoc}}

\title{DATA 5690: Midterm}
\author{Tyler J. Brough}
\date{2024-02-28}

\begin{document}
\maketitle
\ifdefined\Shaded\renewenvironment{Shaded}{\begin{tcolorbox}[frame hidden, interior hidden, borderline west={3pt}{0pt}{shadecolor}, enhanced, breakable, boxrule=0pt, sharp corners]}{\end{tcolorbox}}\fi

\renewcommand*\contentsname{Table of contents}
{
\hypersetup{linkcolor=}
\setcounter{tocdepth}{2}
\tableofcontents
}
\bookmarksetup{startatroot}

\chapter*{\texorpdfstring{\textbf{Introduction}}{Introduction}}\label{introduction}
\addcontentsline{toc}{chapter}{\textbf{Introduction}}

\markboth{\textbf{Introduction}}{\textbf{Introduction}}

This is your midterm exam.

\bookmarksetup{startatroot}

\chapter*{\texorpdfstring{\textbf{Frequentist
Analysis}}{Frequentist Analysis}}\label{frequentist-analysis}
\addcontentsline{toc}{chapter}{\textbf{Frequentist Analysis}}

\markboth{\textbf{Frequentist Analysis}}{\textbf{Frequentist Analysis}}

\textbf{1.} In this question your task is to carry out statistical
inference for a binomial proportion from the frequentist perspective.

The frequentist agent confronts a tootsie roll candy machine with a
fixed but unknown probability of dispensing a cherry tootsie roll
denoted by \(\theta\).

\begin{enumerate}
\def\labelenumi{\alph{enumi}.}
\tightlist
\item
  Generate artificial data for this scenario with the following code:
\end{enumerate}

\begin{Shaded}
\begin{Highlighting}[]
\ImportTok{import}\NormalTok{ numpy }\ImportTok{as}\NormalTok{ np}

\NormalTok{np.random.seed(}\DecValTok{42}\NormalTok{)}

\NormalTok{theta }\OperatorTok{=} \FloatTok{0.30}
\NormalTok{tootsie\_rolls }\OperatorTok{=}\NormalTok{ np.random.binomial(n}\OperatorTok{=}\DecValTok{1}\NormalTok{, p}\OperatorTok{=}\NormalTok{theta, size}\OperatorTok{=}\DecValTok{50}\NormalTok{)}
\end{Highlighting}
\end{Shaded}

The agent is given these data and told they represent draws from the
candy machine where an observation of \(1\) represents a cherry tootsie
roll and an observation of \(0\) represents a vanilla tootsie roll.

\begin{enumerate}
\def\labelenumi{\alph{enumi}.}
\setcounter{enumi}{1}
\item
  Compute the maximum likelihood estimator \(\hat{\theta}_{MLE}\) as
  though you were the agent. What is the agent's numerical point
  estimate of this maximum likelihood estimator?
\item
  What is the sampling distribution of \(\hat{\theta}_{MLE}\) according
  to the \emph{Central Limit Theorem}? Make a plot of it using
  \texttt{matplotlib.pyplot}.
\item
  Compute a \(95\%\) confidence interval for \(\hat{\theta}_{MLE}\).
  What are the upper and lower bounds? Give a formal interpretation of
  this confidence interval.
\item
  Conduct a hypothesis test that the tootsie roll machine is biased
  towards dispensing vanilla tootsie rolls with a level of significance
  of \(5\%\).

  \begin{itemize}
  \tightlist
  \item
    State the null hypothesis.
  \item
    State the alternative hypothesis.
  \item
    Compute the test statistic and report its numerical value.
  \item
    Compute the rejection region and report its numerical value.
  \item
    Is this a one-tailed or two-tailed test?
  \item
    What does the agent conclude? State it formally.
  \end{itemize}
\item
  Please redo parts a-e for \(\theta = 0.45\).
\end{enumerate}

\textbf{2.} In this question your task is to carry out statistical
inference for count data from the frequentist perspective. Assume that
these data represent visitors that arrive per hour to take a turn at the
tootsie roll machine. Let \(\lambda\) be the hourly arrival rate of the
visitors.

\begin{enumerate}
\def\labelenumi{\alph{enumi}.}
\tightlist
\item
  Generate artificial data for this problem with the following code:
\end{enumerate}

\begin{Shaded}
\begin{Highlighting}[]
\ImportTok{import}\NormalTok{ numpy }\ImportTok{as}\NormalTok{ np}

\NormalTok{np.random.seed(}\DecValTok{42}\NormalTok{)}

\NormalTok{lam }\OperatorTok{=} \DecValTok{20}
\NormalTok{visits }\OperatorTok{=}\NormalTok{ np.random.poisson(lam}\OperatorTok{=}\NormalTok{lam, size}\OperatorTok{=}\DecValTok{50}\NormalTok{)}
\end{Highlighting}
\end{Shaded}

\begin{enumerate}
\def\labelenumi{\alph{enumi}.}
\setcounter{enumi}{1}
\item
  The agent doesn't see the data-generating process but assumes that
  they come from a Poisson distribution with a fixed but unknown
  \(\lambda\) parameter. Compute the maximum likelihood estimator
  \(\hat{\lambda}_{MLE}\).
\item
  What is the sampling distribution of \(\hat{\lambda}_{MLE}\) according
  to the \emph{Central Limit Theorem}? Make a plot of it using
  \texttt{matplotlib.pyplot}.
\item
  Compute a \(95\%\) confidence interval for \(\hat{\lambda}_{MLE}\).
  What are the upper and lower bounds? Give a formal interpretation of
  this confidence interval.
\item
  Conduct a hypothesis test that the true arrival rate is 18 visitors
  per hour.

  \begin{itemize}
  \tightlist
  \item
    State the null hypothesis.
  \item
    State the alternative hypothesis.
  \item
    Compute the test statistic and report its numerical value.
  \item
    Compute the rejection region and report its numerical value.
  \item
    Is this a one-tailed or two-tailed test?
  \item
    What does the agent conclude? State it formally.
  \end{itemize}
\end{enumerate}

\textbf{3.} Use the IID bootstrap procedure to generate an approximate
sampling distribution for \(\hat{\lambda}_{MLE}\) in the previous
problem using the same data that were given to the agent.

\begin{enumerate}
\def\labelenumi{\alph{enumi}.}
\tightlist
\item
  You can produce a single bootstrap sample with the following code:
\end{enumerate}

\begin{Shaded}
\begin{Highlighting}[]
\NormalTok{np.random.seed(}\DecValTok{42}\NormalTok{)}

\NormalTok{x\_b }\OperatorTok{=}\NormalTok{ np.random.choice(a}\OperatorTok{=}\NormalTok{x, size}\OperatorTok{=}\DecValTok{50}\NormalTok{, replace}\OperatorTok{=}\VariableTok{True}\NormalTok{)}
\end{Highlighting}
\end{Shaded}

Given this bootstrap sample you would then compute a bootstrap
replication of the MLE: \(\hat{\lambda}^{b}_{MLE}\).

\begin{enumerate}
\def\labelenumi{\alph{enumi}.}
\setcounter{enumi}{1}
\item
  Repeat the above for \(b = 1, \ldots, B\) with \(B = 10,000\).
\item
  Reproduce the confidence interval and hypothesis test from question 2
  above but using the bootstrap sampling distribution rather than
  appealing to the CLT.
\item
  Compare this computational procedure to the classical approach using
  the CLT.
\end{enumerate}

\bookmarksetup{startatroot}

\chapter*{\_\_Bayesian Analysis
\{.unnumbered\}}\label{bayesian-analysis-.unnumbered}
\addcontentsline{toc}{chapter}{\_\_Bayesian Analysis \{.unnumbered\}}

\markboth{\_\_Bayesian Analysis \{.unnumbered\}}{\_\_Bayesian Analysis
\{.unnumbered\}}

\textbf{4.} Reproduce the statistical inference for the data from
problem 1 above but from the subjective Bayesian perspective. - Assume
the agent has a prior of \(\theta \sim Beta(a=1, b=1)\). - Compute the
posterior distribution. - Make plots of the prior, likelihood and
posterior using \texttt{matplotlib.pyplot}. - Calculate the posterior
probability that \(\theta = 0.5\). - Compute a \(95\%\) equal-tailed
credibility interval. - Using Bayes' factors conduct a hypothesis test
for \(H_{1}: \theta = 0.5\) (i.e.~a fair coin) against
\(H_{2}: \theta \ne 0.5\) (i.e.~a biased coin). See Clyde, Merlise and
Çetinkaya-Rundel, Mine and Rundel, Colin and Banks, David and Chai,
Christine and Huang, Lizzy (2022) Chapter 3 for details on implementing
Bayes' factors. - Interpret the results. Compare the results to the
frequentist procedure.

\textbf{5.} Reproduce the statistical inference for the data from
problem 2 above but from the subjective Bayesian perspective. - Assume
the agent has the prior: \(\lambda \sim Gamma(\alpha, \beta)\), which is
the conjugate prior for the Poisson likelihood function. - Compute the
posterior distribution. - Make plots of the prior, likelihood, and
posterior using \texttt{matplotlib.pyplot}. - Compute a \(95\%\)
equal-tailed credibility interval. - Using Bayes' factors conduct a
hypothesis test for \(H_{1}: \lambda = 18\) against
\(H_{2}: \lambda \ne 18\). Use a diffuse prior for \(H_{2}\).

\bookmarksetup{startatroot}

\chapter*{References}\label{references}
\addcontentsline{toc}{chapter}{References}

\markboth{References}{References}

\bookmarksetup{startatroot}

\chapter*{References}\label{references-1}
\addcontentsline{toc}{chapter}{References}

\markboth{References}{References}

\phantomsection\label{refs}
\begin{CSLReferences}{1}{0}
\bibitem[\citeproctext]{ref-ClydeEtAl2022}
Clyde, Merlise and Çetinkaya-Rundel, Mine and Rundel, Colin and Banks,
David and Chai, Christine and Huang, Lizzy. 2022. \emph{{An Introduction
to Bayesian Thinking}: {A Companion to the Statistics with R Course}}.
\url{https://statswithr.github.io/book/}.

\end{CSLReferences}



\end{document}
